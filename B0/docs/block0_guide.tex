\documentclass[11pt]{article}
\usepackage[a4paper, margin=2.5cm]{geometry}
\usepackage{hyperref}
\usepackage{enumitem}
\usepackage{xcolor}
\usepackage{titlesec}

\definecolor{satblue}{RGB}{30, 60, 120}
\titleformat{\section}{\large\bfseries\color{satblue}}{\thesection}{1em}{}
\titleformat{\subsection}{\normalsize\bfseries}{\thesubsection}{1em}{}
\setlist[itemize]{noitemsep, topsep=3pt}
\setlist[enumerate]{noitemsep, topsep=3pt}

\title{\textbf{S-ART Training Program}\\Bloque 0 — Fundamentos y Herramientas}
\author{Student Astronomical Radio Telescope (S-ART)}
\date{}

\begin{document}
\maketitle

\section*{Cómo usar esta guía}
Sigue las secciones B0.1, B0.2 y B0.3 en orden.
Todas las entregas se realizan por Pull Request (PR) y deben ser reproducibles.

\section{B0.1 — Introducción a CubeSats}
Ver instrucciones en: \texttt{B0/docs/cubesat\_notes/README.md}

\section{B0.2 — Git y GitHub aplicado a S-ART}
Runbook: \texttt{B0/docs/runbooks/rb\_b0\_2\_git\_pr.md}

\section{B0.3 — Análisis de telemetría (problema guiado)}
Runbook: \texttt{B0/docs/runbooks/rb\_b0\_3\_tm.md}

\end{document}
